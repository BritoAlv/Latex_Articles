% add --shell-escape to pdflatex arguments.
% add following key to have keyboard shortcuts
%{
%    "key": "shift+b",
%    "command": "commandId",
%    "when": "editorTextFocus"
%},
%{
%"key": "shift+B",
%"command": "editor.action.insertSnippet",
%"when": "editorLangId == latex && editorTextFocus",
%"args": {
%    "snippet": "\\textbf{${TM_SELECTED_TEXT}$0}"
%}
%}

\documentclass[14pt]{extarticle}
\usepackage[left=2cm , right = 2cm, top=2cm]{geometry}
\usepackage{helvet}
\usepackage{parskip}
\usepackage{amsmath}
\usepackage{amssymb}
\usepackage{graphicx}
\usepackage[spanish]{babel}
\usepackage[dvipsnames]{xcolor}
\usepackage{tcolorbox} % above of the svg package
\usepackage{svg} 
\usepackage{hyperref}
\usepackage{minted}
\renewcommand{\sfdefault}{lmss}  % este activa la letra lmss
\renewcommand{\familydefault}{\sfdefault} % este activa la letra lmss
\sffamily % este activa la letra lmss
%\hyperlink{page.2}{Go to page 2}
%\newpage
%text on page 2
%\begin{figure}[htbp]
%  \centering
%  \includesvg{plot.svg}
%  \caption{svg image}
%\end{figure}

%\begin{minted}{csharp}
%    // single comment
%    \end{minted}

%\begin{align}
%    \frac{d}{dx} \ln x &= \lim_{h\to 0} \frac{\ln(x+h) - \ln x}{h} \\
%    &= \ln e^{1/x} &&\text{How this follows is left as an exercise.}\\
%    &= \frac{1}{x} &&\text{Using the definition of ln as inverse function}
%   \end{align}


\begin{document}



@BritoAlv



\begin{tcolorbox}[colback=blue!5!white,colframe=blue!75!black, title = Curvas de Segundo Grado]

La ecuación general de una curva de segundo grado es de la forma: $$Ax^2+Bxy+Cy^2+Dx+Ey+F = 0$$
\end{tcolorbox}

\section{Rotación y Traslación}

\subsection{Translación}

Si sustituimos $x = x' + h$, $y = y'+k$, (trasladar el eje de coordenadas al punto $(h;k)$) obtenemos las siguientes relaciones dado que nuestro objetivo es eliminar los coeficientes $D,E$: 

$$
2Ah + Bk + D = 0  \, \, \, \, \, \, \, \, \, \, \, \, Bh + 2Ck + E = 0
$$

Esto es un $2*2$ con respecto a $h,k$, que se puede escribir de la siguiente forma: 

$$
\begin{bmatrix}
    2A & B \\ B & 2C
\end{bmatrix}
    \begin{bmatrix}
        h \\ k
    \end{bmatrix} = 
    \begin{bmatrix}
        -D \\ -E
    \end{bmatrix}
$$ 

Que significa que si $B^2 -4AC \neq 0$, podemos encontrar soluciones para $h,k$ y entonces cancelar los términos lineales, la nueva ecuación sería:

$$Ax'^2 + Bx'y' + Cy'^2 + R = 0$$

Con respecto a la ecuación que teníamos antes $D$ cambió, pero notemos que $A,B,C$ se mantienen invariantes. En el caso de que $B^2-4AC$ es $0$ no podemos eliminar ambos términos siempre, por lo que este método no es válido, sin embargo podemos realizar el siguiente razonamiento:

Como $B^2 = 4AC$ $A,C$ han de tener el mismo signo, sin pérdida de generalidad asumamos que $A,C \geq 0$ , entonces podemos escribir la ecuación de la forma:

$$(\sqrt{A}x + \sqrt{C}y)^2+Dx+Ey+F = 0$$

Si exactamente uno de $A,C$ es $0$, entonces después de completar el cuadrado obtenemos la ecuación de una parabola, si ambos son $0$ tenemos la ecuación de una línea, en el caso en que ambos son distintos de $0$, lo dejaremos para cuando analize la rotación.

\subsection{Rotación}

El resultado final de la rotación es eliminar de la ecuación general el término mixto, $xy$, para aplicar una rotación de un ángulo $\theta$ hacemos lo siguiente:

$$
x = x'\cos(\theta) - y'\sin(\theta)  \, \, \, \, \, \, \, \, \, \, \, \,  y = x'\sin(\theta)+y'\cos(\theta)
$$

Que no es más que el resultado de aplicar la transformación lineal indicada por la matriz de rotación, el resultado final de esto es una ecuación general de la forma:

$$Ax'^2+Cy'^2+Mx+Ny+F = 0$$

Donde el ángulo de rotación es obtenido a partir de $\tan \theta = \frac{B}{A-C}$, si $A = C$ el ángulo es $45$,notemos si primero aplicamos una Translación y no fue posible eliminar ambos términos lineales, entonces la ecuación que nos queda después de rotarla no poseerá término mixto por lo que al menos uno de los $A', C'$ obtenido será $0$ y estaremos en presencia de un parábola, en cambio si fue posible eliminar ambos términos lineares y después aplicamos la rotación tendremos una ecuación general de la forma:

$$Tx^2 + Py^2 + O = 0 $$, que en dependencia de los signos de $T,P,O$ representará una elipse  o  una hipérbola en el caso no-degenerado. Si $T,P$ tienen el mismo signo contrario al de $O$ es una elipse, si tienen distinto signo es una hiperbola.

\subsection{Conclusión}

A partir de rotaciones y translaciones fue posible llevar la ecuación general a las formas particulares de cada cónica dependiendo de los parámetros de la ecuación.

\section{Asimptotas de una hiperbola}


\begin{tcolorbox}[colback=blue!5!white,colframe=blue!75!black, title = Motivación]

    Demostrar que las rectas representadas por $$Ax^2 + 2Bxy + Cy^2 = 0$$ son asíntotas de la curva
dada por la ecuación $$Ax^2 + 2Bxy + Cy^2 + F = 0$$

\end{tcolorbox}

\begin{tcolorbox}[colback=red!5!white,colframe=red!75!black, title = Solución]

    Primero demostraré que la expresión $Ax^2 + 2Bxy + Cy^2 = 0$ representa a lo más rectas en el plano, si aplicamos una rotación a la expresión anterior tendremos una expresión de la forma: $A'x'2+C'y'^2 = 0$, que en dependencia de los signos de $A', C'$ representa un punto, una recta o dos rectas o nada.

    Análogamente si aplicamos una rotación a la segunda expresión obtenemos $A'x'2+C'y'2 + F = 0$, por otro lado tenemos la siguiente invariante (propiedad de la rotación): 
    $$
        (2B)^2-4AC = (B')^2-4A'C' = -4A'C'
    $$

    Asumiré que $F \neq 0$, además notemos que si uno de $A', C'$ es $0$ el otro también lo es, consecuentemente $B^2 = AC$ y la segunda expresión no representa una curva, representa a lo más dos rectas. Por lo que asumiré que ninguno es $0$. Por otro lado si la curva es una elipse entonces se tiene que $A',C'$ tienen el mismo signo y por tanto la única solución en la ecuación de la asímptota es $(0,0)$ que no es una recta, por lo que concluyo que $A',C'$ tienen distinto signo y por tanto es una hiperbola. A partir de este momento el problema se reduce a demostrar que las asímptotas de una hiperbola con ecuación:
    
$$\frac{x^2}{a^2} - \frac{y^2}{b^2} = f$$ Son las rectas $y = \pm \frac{b}{a}x$
\end{tcolorbox}

\newpage

Sin pérdida de generalidad puedo asumir que $f = 1$, para calcular las asímptotas de la hiperbola escribamos la ecuación de la forma:

$$y = \pm \frac{b}{a}x \sqrt{1 - \frac{a^2}{x^2}}$$

Cuando $x$ tiende a infinito los valores de la $y$ tienden a $\pm \frac{b}{a}x$, esto demuestra que los candidatos a ser asimptotas son las ecuaciones $y = \pm \frac{b}{a}x$ falta demostrar que $\lim_{x \to \infty} y \pm \frac{b}{a}x = $ constante, esta constante se puede demostrar que es $0$, por lo que ambas rectas son asímptotas de la hiperbola.

\end{document}

