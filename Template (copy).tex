\documentclass[14pt]{extarticle}
\usepackage[left=2cm , right = 2cm, top=2cm]{geometry}
\usepackage{helvet}
\usepackage{parskip}
\usepackage{amsmath}
\usepackage{amssymb}
\usepackage{graphicx}
\usepackage[spanish]{babel}
\usepackage[dvipsnames]{xcolor}
\usepackage{tcolorbox}
\usepackage{hyperref}
\renewcommand{\sfdefault}{lmss}  % este activa la letra lmss
\renewcommand{\familydefault}{\sfdefault} % este activa la letra lmss
\sffamily % este activa la letra lmss
\begin{document}

@BritoAlv


El humanismo fue un movimiento filosófico, intelectual y cultural que comenzó
en el siglo XIV en Italia. Fue unas de las causas del surgimiento
de la Edad Moderna, se enfocaba en la capacidad de los seres humanos de
conocer el mundo e interactuar con el mediante la razón. Surgió ante la
sociedad burguesa, reemplazó la concepción teocéntrica del universo propio
del mundo medieval por una forma de pensamiento antropocéntrica. Los
humanistas proponían desarrollar un pensamiento crítico en lugar del
pensamiento dogmático que explicaba el mundo a través de la relación divina.
Consideraban al ser humano como la más perfecta creación de Dios y afirmaba
que la naturaleza estaba subordinada a lo humano. Retomó los valore estéticos
y filosóficos de la cultura clásica (Antigua Grecia y Roma) centrados en las 
manifestaciones humanas más que en las divinas. Recuperaron las antiguas obras
filosóficas y las estudiaron en sus lenguas originales. Pretendieron usar la 
razón y la experiencia en vez de la fé para conocer y explicar el mundo. 
Dio lugar al renacimiento como cultura del humanismo. Se destacan Nicolas
Maquiavelo (), Tomas Moro, Nicolas Cosa.

El humanismo dice que el ser humano es el centro de todas las cosas mientras
que la religión dice que Dios es el centro de todo, es la adoración o 
reconocimiento de la afirmación del hombre tenga derecho a la soberanía.

En lugar de ello sostuvo que Dios busca impedir la auto-realización del hombre,
el hombre debe ser su propio señor y soberano escogiendo lo bueno y lo malo en
términos de su propio auto-interés.

El hombre es importante y su inteligencia un valor superior al igual que la razón humana.

- pacifismo
- lógica aristotélica

Si el ser humano es racional y la realidad también entonces es posible conformar
a este mundo en que vivimos tanto en su faceta natural como en lo social de
acuerdo q nuestra conveniencia e intereses. Se deja de un lado la visión cristiana que predominó
en la edad media en la cual se destacaba el desprecio por el mundo del más acá y 
por el hombre córporeo para dar paso a una nueva valoración del hombre como ser que
piensa y siente. Se reconoce el valor del hombre como ser terrestre, inserto en el
mundo de la naturalea y la historia y capaz de forjar en él mismo su propio destino. 
Es así que para la filosofía moderna el hombre y el mundo sensible son punto de partida para toda reflexión y conocimiento.


\end{document}

